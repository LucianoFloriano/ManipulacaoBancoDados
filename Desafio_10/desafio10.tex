% Options for packages loaded elsewhere
\PassOptionsToPackage{unicode}{hyperref}
\PassOptionsToPackage{hyphens}{url}
%
\documentclass[
]{article}
\usepackage{amsmath,amssymb}
\usepackage{iftex}
\ifPDFTeX
  \usepackage[T1]{fontenc}
  \usepackage[utf8]{inputenc}
  \usepackage{textcomp} % provide euro and other symbols
\else % if luatex or xetex
  \usepackage{unicode-math} % this also loads fontspec
  \defaultfontfeatures{Scale=MatchLowercase}
  \defaultfontfeatures[\rmfamily]{Ligatures=TeX,Scale=1}
\fi
\usepackage{lmodern}
\ifPDFTeX\else
  % xetex/luatex font selection
\fi
% Use upquote if available, for straight quotes in verbatim environments
\IfFileExists{upquote.sty}{\usepackage{upquote}}{}
\IfFileExists{microtype.sty}{% use microtype if available
  \usepackage[]{microtype}
  \UseMicrotypeSet[protrusion]{basicmath} % disable protrusion for tt fonts
}{}
\makeatletter
\@ifundefined{KOMAClassName}{% if non-KOMA class
  \IfFileExists{parskip.sty}{%
    \usepackage{parskip}
  }{% else
    \setlength{\parindent}{0pt}
    \setlength{\parskip}{6pt plus 2pt minus 1pt}}
}{% if KOMA class
  \KOMAoptions{parskip=half}}
\makeatother
\usepackage{xcolor}
\usepackage[margin=1in]{geometry}
\usepackage{color}
\usepackage{fancyvrb}
\newcommand{\VerbBar}{|}
\newcommand{\VERB}{\Verb[commandchars=\\\{\}]}
\DefineVerbatimEnvironment{Highlighting}{Verbatim}{commandchars=\\\{\}}
% Add ',fontsize=\small' for more characters per line
\usepackage{framed}
\definecolor{shadecolor}{RGB}{248,248,248}
\newenvironment{Shaded}{\begin{snugshade}}{\end{snugshade}}
\newcommand{\AlertTok}[1]{\textcolor[rgb]{0.94,0.16,0.16}{#1}}
\newcommand{\AnnotationTok}[1]{\textcolor[rgb]{0.56,0.35,0.01}{\textbf{\textit{#1}}}}
\newcommand{\AttributeTok}[1]{\textcolor[rgb]{0.13,0.29,0.53}{#1}}
\newcommand{\BaseNTok}[1]{\textcolor[rgb]{0.00,0.00,0.81}{#1}}
\newcommand{\BuiltInTok}[1]{#1}
\newcommand{\CharTok}[1]{\textcolor[rgb]{0.31,0.60,0.02}{#1}}
\newcommand{\CommentTok}[1]{\textcolor[rgb]{0.56,0.35,0.01}{\textit{#1}}}
\newcommand{\CommentVarTok}[1]{\textcolor[rgb]{0.56,0.35,0.01}{\textbf{\textit{#1}}}}
\newcommand{\ConstantTok}[1]{\textcolor[rgb]{0.56,0.35,0.01}{#1}}
\newcommand{\ControlFlowTok}[1]{\textcolor[rgb]{0.13,0.29,0.53}{\textbf{#1}}}
\newcommand{\DataTypeTok}[1]{\textcolor[rgb]{0.13,0.29,0.53}{#1}}
\newcommand{\DecValTok}[1]{\textcolor[rgb]{0.00,0.00,0.81}{#1}}
\newcommand{\DocumentationTok}[1]{\textcolor[rgb]{0.56,0.35,0.01}{\textbf{\textit{#1}}}}
\newcommand{\ErrorTok}[1]{\textcolor[rgb]{0.64,0.00,0.00}{\textbf{#1}}}
\newcommand{\ExtensionTok}[1]{#1}
\newcommand{\FloatTok}[1]{\textcolor[rgb]{0.00,0.00,0.81}{#1}}
\newcommand{\FunctionTok}[1]{\textcolor[rgb]{0.13,0.29,0.53}{\textbf{#1}}}
\newcommand{\ImportTok}[1]{#1}
\newcommand{\InformationTok}[1]{\textcolor[rgb]{0.56,0.35,0.01}{\textbf{\textit{#1}}}}
\newcommand{\KeywordTok}[1]{\textcolor[rgb]{0.13,0.29,0.53}{\textbf{#1}}}
\newcommand{\NormalTok}[1]{#1}
\newcommand{\OperatorTok}[1]{\textcolor[rgb]{0.81,0.36,0.00}{\textbf{#1}}}
\newcommand{\OtherTok}[1]{\textcolor[rgb]{0.56,0.35,0.01}{#1}}
\newcommand{\PreprocessorTok}[1]{\textcolor[rgb]{0.56,0.35,0.01}{\textit{#1}}}
\newcommand{\RegionMarkerTok}[1]{#1}
\newcommand{\SpecialCharTok}[1]{\textcolor[rgb]{0.81,0.36,0.00}{\textbf{#1}}}
\newcommand{\SpecialStringTok}[1]{\textcolor[rgb]{0.31,0.60,0.02}{#1}}
\newcommand{\StringTok}[1]{\textcolor[rgb]{0.31,0.60,0.02}{#1}}
\newcommand{\VariableTok}[1]{\textcolor[rgb]{0.00,0.00,0.00}{#1}}
\newcommand{\VerbatimStringTok}[1]{\textcolor[rgb]{0.31,0.60,0.02}{#1}}
\newcommand{\WarningTok}[1]{\textcolor[rgb]{0.56,0.35,0.01}{\textbf{\textit{#1}}}}
\usepackage{graphicx}
\makeatletter
\def\maxwidth{\ifdim\Gin@nat@width>\linewidth\linewidth\else\Gin@nat@width\fi}
\def\maxheight{\ifdim\Gin@nat@height>\textheight\textheight\else\Gin@nat@height\fi}
\makeatother
% Scale images if necessary, so that they will not overflow the page
% margins by default, and it is still possible to overwrite the defaults
% using explicit options in \includegraphics[width, height, ...]{}
\setkeys{Gin}{width=\maxwidth,height=\maxheight,keepaspectratio}
% Set default figure placement to htbp
\makeatletter
\def\fps@figure{htbp}
\makeatother
\setlength{\emergencystretch}{3em} % prevent overfull lines
\providecommand{\tightlist}{%
  \setlength{\itemsep}{0pt}\setlength{\parskip}{0pt}}
\setcounter{secnumdepth}{-\maxdimen} % remove section numbering
\ifLuaTeX
  \usepackage{selnolig}  % disable illegal ligatures
\fi
\usepackage{bookmark}
\IfFileExists{xurl.sty}{\usepackage{xurl}}{} % add URL line breaks if available
\urlstyle{same}
\hypersetup{
  pdftitle={desafio10},
  pdfauthor={Luciano Floriano},
  hidelinks,
  pdfcreator={LaTeX via pandoc}}

\title{desafio10}
\author{Luciano Floriano}
\date{2025-10-02}

\begin{document}
\maketitle

\begin{verbatim}
## Using virtual environment "C:/Users/luflo/AppData/Local/R/cache/R/reticulate/uv/cache/archive-v0/qj_0-rP_hm7QfbZg_7cPu" ...
\end{verbatim}

\begin{verbatim}
## + "C:/Users/luflo/AppData/Local/R/cache/R/reticulate/uv/cache/archive-v0/qj_0-rP_hm7QfbZg_7cPu/Scripts/python.exe" -m pip install --upgrade --no-user polars
\end{verbatim}

\begin{verbatim}
## Using virtual environment "C:/Users/luflo/AppData/Local/R/cache/R/reticulate/uv/cache/archive-v0/qj_0-rP_hm7QfbZg_7cPu" ...
\end{verbatim}

\begin{verbatim}
## + "C:/Users/luflo/AppData/Local/R/cache/R/reticulate/uv/cache/archive-v0/qj_0-rP_hm7QfbZg_7cPu/Scripts/python.exe" -m pip install --upgrade --no-user fastexcel
\end{verbatim}

\begin{verbatim}
## Using virtual environment "C:/Users/luflo/AppData/Local/R/cache/R/reticulate/uv/cache/archive-v0/qj_0-rP_hm7QfbZg_7cPu" ...
\end{verbatim}

\begin{verbatim}
## + "C:/Users/luflo/AppData/Local/R/cache/R/reticulate/uv/cache/archive-v0/qj_0-rP_hm7QfbZg_7cPu/Scripts/python.exe" -m pip install --upgrade --no-user pyarrow
\end{verbatim}

Data de realização desse código:

\begin{Shaded}
\begin{Highlighting}[]
\ImportTok{import}\NormalTok{ datetime}

\CommentTok{\# Data e hora atual}
\NormalTok{agora }\OperatorTok{=}\NormalTok{ datetime.datetime.now()}
\BuiltInTok{print}\NormalTok{(}\SpecialStringTok{f"Arquivo compilado em: }\SpecialCharTok{\{}\NormalTok{agora}\SpecialCharTok{.}\NormalTok{strftime(}\StringTok{\textquotesingle{}}\SpecialCharTok{\%d}\StringTok{/\%m/\%Y às \%H:\%M:\%S\textquotesingle{}}\NormalTok{)}\SpecialCharTok{\}}\SpecialStringTok{"}\NormalTok{)}
\end{Highlighting}
\end{Shaded}

\begin{verbatim}
## Arquivo compilado em: 02/10/2025 às 11:20:09
\end{verbatim}

Importando os dados:

\begin{Shaded}
\begin{Highlighting}[]
\ImportTok{import}\NormalTok{ polars }\ImportTok{as}\NormalTok{ pl}

\CommentTok{\# Lê um CSV, mas já traz só as colunas que interessam (código, cidade e estado)}
\NormalTok{aeroportos }\OperatorTok{=}\NormalTok{ pl.read\_csv(}\StringTok{"airports.csv"}\NormalTok{,}
\NormalTok{                         columns}\OperatorTok{=}\NormalTok{[}\StringTok{"IATA\_CODE"}\NormalTok{, }\StringTok{"CITY"}\NormalTok{, }\StringTok{"STATE"}\NormalTok{])}

\NormalTok{aeroportos.head(}\DecValTok{2}\NormalTok{) }\CommentTok{\#mostra só as 2 primeiras linhas.}
\end{Highlighting}
\end{Shaded}

\begin{verbatim}
## shape: (2, 3)
## ┌───────────┬───────────┬───────┐
## │ IATA_CODE ┆ CITY      ┆ STATE │
## │ ---       ┆ ---       ┆ ---   │
## │ str       ┆ str       ┆ str   │
## ╞═══════════╪═══════════╪═══════╡
## │ ABE       ┆ Allentown ┆ PA    │
## │ ABI       ┆ Abilene   ┆ TX    │
## └───────────┴───────────┴───────┘
\end{verbatim}

\begin{Shaded}
\begin{Highlighting}[]
\ImportTok{import}\NormalTok{ polars }\ImportTok{as}\NormalTok{ pl}

\NormalTok{wdi }\OperatorTok{=}\NormalTok{ pl.read\_excel(}\StringTok{"WDIEXCEL.xlsx"}\NormalTok{, sheet\_name}\OperatorTok{=}\StringTok{"Country"}\NormalTok{, }
\NormalTok{                    columns}\OperatorTok{=}\NormalTok{[}\StringTok{"Short Name"}\NormalTok{, }\StringTok{"Region"}\NormalTok{]) }\CommentTok{\#Puxa a aba "Country" e só pega as colunas “nome curto” e “região”.}

\NormalTok{wdi.head(}\DecValTok{2}\NormalTok{)}
\end{Highlighting}
\end{Shaded}

\begin{verbatim}
## shape: (2, 2)
## ┌─────────────┬───────────────────────────┐
## │ Short Name  ┆ Region                    │
## │ ---         ┆ ---                       │
## │ str         ┆ str                       │
## ╞═════════════╪═══════════════════════════╡
## │ Aruba       ┆ Latin America & Caribbean │
## │ Afghanistan ┆ South Asia                │
## └─────────────┴───────────────────────────┘
\end{verbatim}

Um Exemplo Simples

\begin{Shaded}
\begin{Highlighting}[]
\NormalTok{df }\OperatorTok{=}\NormalTok{ pl.DataFrame(\{}
    \StringTok{"grupo"}\NormalTok{: [}\StringTok{"A"}\NormalTok{, }\StringTok{"A"}\NormalTok{, }\StringTok{"B"}\NormalTok{, }\StringTok{"B"}\NormalTok{, }\StringTok{"C"}\NormalTok{],}
    \StringTok{"valor1"}\NormalTok{: [}\DecValTok{10}\NormalTok{, }\DecValTok{15}\NormalTok{, }\DecValTok{10}\NormalTok{, }\VariableTok{None}\NormalTok{, }\DecValTok{25}\NormalTok{],}
    \StringTok{"valor2"}\NormalTok{: [}\DecValTok{5}\NormalTok{, }\VariableTok{None}\NormalTok{, }\DecValTok{20}\NormalTok{, }\DecValTok{30}\NormalTok{, }\VariableTok{None}\NormalTok{] }\CommentTok{\#tabela com grupos e duas colunas de valores (alguns nulos).}
\NormalTok{\})}
\NormalTok{df}
\end{Highlighting}
\end{Shaded}

\begin{verbatim}
## shape: (5, 3)
## ┌───────┬────────┬────────┐
## │ grupo ┆ valor1 ┆ valor2 │
## │ ---   ┆ ---    ┆ ---    │
## │ str   ┆ i64    ┆ i64    │
## ╞═══════╪════════╪════════╡
## │ A     ┆ 10     ┆ 5      │
## │ A     ┆ 15     ┆ null   │
## │ B     ┆ 10     ┆ 20     │
## │ B     ┆ null   ┆ 30     │
## │ C     ┆ 25     ┆ null   │
## └───────┴────────┴────────┘
\end{verbatim}

Operando em valor1

\begin{Shaded}
\begin{Highlighting}[]
\NormalTok{df[}\StringTok{"valor1"}\NormalTok{]}
\end{Highlighting}
\end{Shaded}

\begin{verbatim}
## shape: (5,)
## Series: 'valor1' [i64]
## [
##  10
##  15
##  10
##  null
##  25
## ]
\end{verbatim}

\begin{Shaded}
\begin{Highlighting}[]
\BuiltInTok{print}\NormalTok{(}\StringTok{"Média do valor 1:"}\NormalTok{,(df[}\StringTok{"valor1"}\NormalTok{].mean())) }\CommentTok{\#Mostra a coluna valor1 e já calcula a média dela.}
\end{Highlighting}
\end{Shaded}

\begin{verbatim}
## Média do valor 1: 15.0
\end{verbatim}

Operando em valor1

\begin{Shaded}
\begin{Highlighting}[]
\NormalTok{df.select([}
\NormalTok{  pl.col(}\StringTok{"valor1"}\NormalTok{).mean().alias(}\StringTok{"media\_v1"}\NormalTok{),}
\NormalTok{  pl.col(}\StringTok{"valor2"}\NormalTok{).mean() }\CommentTok{\#retorna um mini{-}DataFrame com as médias.}
\NormalTok{])}
\end{Highlighting}
\end{Shaded}

\begin{verbatim}
## shape: (1, 2)
## ┌──────────┬───────────┐
## │ media_v1 ┆ valor2    │
## │ ---      ┆ ---       │
## │ f64      ┆ f64       │
## ╞══════════╪═══════════╡
## │ 15.0     ┆ 18.333333 │
## └──────────┴───────────┘
\end{verbatim}

Quais são as médias da variável valor1 e o valor mínimo da variável
valor2 para cada um dos grupos definidos por grupo?

\begin{Shaded}
\begin{Highlighting}[]
\NormalTok{df.group\_by(}\StringTok{"grupo"}\NormalTok{).agg([}
\NormalTok{  pl.col(}\StringTok{"valor1"}\NormalTok{).mean().alias(}\StringTok{"media\_valor1"}\NormalTok{),}
\NormalTok{  pl.col(}\StringTok{"valor2"}\NormalTok{).}\BuiltInTok{min}\NormalTok{().alias(}\StringTok{"min\_valor2"}\NormalTok{)}
\NormalTok{]).sort(}\StringTok{"grupo"}\NormalTok{) }\CommentTok{\#Calcula a média de valor1 e o mínimo de valor2 para cada grupo A, B, C. Depois organiza pelo nome do grupo.}
\end{Highlighting}
\end{Shaded}

\begin{verbatim}
## shape: (3, 3)
## ┌───────┬──────────────┬────────────┐
## │ grupo ┆ media_valor1 ┆ min_valor2 │
## │ ---   ┆ ---          ┆ ---        │
## │ str   ┆ f64          ┆ i64        │
## ╞═══════╪══════════════╪════════════╡
## │ A     ┆ 12.5         ┆ 5          │
## │ B     ┆ 10.0         ┆ 20         │
## │ C     ┆ 25.0         ┆ null       │
## └───────┴──────────────┴────────────┘
\end{verbatim}

Calcule o percentual de vôos das cias. aéreas ``AA'' e ``DL'' que
atrasaram pelo menos 30 minutos nas chegadas aos aeroportos ``SEA'',
``MIA'' e ``BWI''.

\begin{Shaded}
\begin{Highlighting}[]
\ImportTok{import}\NormalTok{ polars }\ImportTok{as}\NormalTok{ pl}

\CommentTok{\# Correção: usar schema\_overrides e ler arquivo ZIP corretamente}
\NormalTok{voos }\OperatorTok{=}\NormalTok{ pl.read\_csv(}\StringTok{"flights.csv"}\NormalTok{,}
\NormalTok{                   columns}\OperatorTok{=}\NormalTok{[}\StringTok{"AIRLINE"}\NormalTok{, }\StringTok{"ARRIVAL\_DELAY"}\NormalTok{, }\StringTok{"DESTINATION\_AIRPORT"}\NormalTok{],}
\NormalTok{                   schema\_overrides}\OperatorTok{=}\NormalTok{\{}\StringTok{"AIRLINE"}\NormalTok{: pl.Utf8,}
                                    \StringTok{"ARRIVAL\_DELAY"}\NormalTok{: pl.Int32,}
                                    \StringTok{"DESTINATION\_AIRPORT"}\NormalTok{: pl.Utf8\}) }\CommentTok{\#Carrega só as colunas úteis (companhia, atraso e destino). Também define os tipos de dados certinhos.}
\NormalTok{voos.shape}
\end{Highlighting}
\end{Shaded}

\begin{verbatim}
## (5819079, 3)
\end{verbatim}

\begin{Shaded}
\begin{Highlighting}[]
\NormalTok{voos.head(}\DecValTok{3}\NormalTok{) }
\end{Highlighting}
\end{Shaded}

\begin{verbatim}
## shape: (3, 3)
## ┌─────────┬─────────────────────┬───────────────┐
## │ AIRLINE ┆ DESTINATION_AIRPORT ┆ ARRIVAL_DELAY │
## │ ---     ┆ ---                 ┆ ---           │
## │ str     ┆ str                 ┆ i32           │
## ╞═════════╪═════════════════════╪═══════════════╡
## │ AS      ┆ SEA                 ┆ -22           │
## │ AA      ┆ PBI                 ┆ -9            │
## │ US      ┆ CLT                 ┆ 5             │
## └─────────┴─────────────────────┴───────────────┘
\end{verbatim}

Calcule o percentual de vôos das cias. aéreas ``AA'' e ``DL'' que
atrasaram pelo menos 30 minutos nas chegadas aos aeroportos ``SEA'',
``MIA'' e ``BWI''.

\begin{Shaded}
\begin{Highlighting}[]
\NormalTok{resultado }\OperatorTok{=}\NormalTok{ (}
\NormalTok{  voos.drop\_nulls([}\StringTok{"AIRLINE"}\NormalTok{, }\StringTok{"DESTINATION\_AIRPORT"}\NormalTok{, }\StringTok{"ARRIVAL\_DELAY"}\NormalTok{])}
\NormalTok{  .}\BuiltInTok{filter}\NormalTok{(}
\NormalTok{    pl.col(}\StringTok{"AIRLINE"}\NormalTok{).is\_in([}\StringTok{"AA"}\NormalTok{, }\StringTok{"DL"}\NormalTok{]) }\OperatorTok{\&}
\NormalTok{    pl.col(}\StringTok{"DESTINATION\_AIRPORT"}\NormalTok{).is\_in([}\StringTok{"SEA"}\NormalTok{, }\StringTok{"MIA"}\NormalTok{, }\StringTok{"BWI"}\NormalTok{])}
\NormalTok{    )}
\NormalTok{    .group\_by([}\StringTok{"AIRLINE"}\NormalTok{, }\StringTok{"DESTINATION\_AIRPORT"}\NormalTok{])}
\NormalTok{    .agg([}
\NormalTok{      (pl.col(}\StringTok{"ARRIVAL\_DELAY"}\NormalTok{) }\OperatorTok{\textgreater{}} \DecValTok{30}\NormalTok{).mean().alias(}\StringTok{"atraso\_medio"}\NormalTok{)}
\NormalTok{      ])}
\NormalTok{) }\CommentTok{\#Filtra só voos da AA e DL para os aeroportos pedidos, remove nulos e calcula o percentual que atrasou mais de 30 minutos.}
\NormalTok{resultado.sort(}\StringTok{"atraso\_medio"}\NormalTok{)}
\end{Highlighting}
\end{Shaded}

\begin{verbatim}
## shape: (6, 3)
## ┌─────────┬─────────────────────┬──────────────┐
## │ AIRLINE ┆ DESTINATION_AIRPORT ┆ atraso_medio │
## │ ---     ┆ ---                 ┆ ---          │
## │ str     ┆ str                 ┆ f64          │
## ╞═════════╪═════════════════════╪══════════════╡
## │ DL      ┆ BWI                 ┆ 0.069455     │
## │ DL      ┆ SEA                 ┆ 0.072967     │
## │ DL      ┆ MIA                 ┆ 0.090467     │
## │ AA      ┆ MIA                 ┆ 0.117894     │
## │ AA      ┆ SEA                 ┆ 0.124212     │
## │ AA      ┆ BWI                 ┆ 0.127523     │
## └─────────┴─────────────────────┴──────────────┘
\end{verbatim}

\subsection{Segundo conjunto de
exercícios}\label{segundo-conjunto-de-exercuxedcios}

Dados Clientes

\begin{Shaded}
\begin{Highlighting}[]
\CommentTok{\# Criando DataFrames}
\NormalTok{clientes }\OperatorTok{=}\NormalTok{ pl.DataFrame(\{}
    \StringTok{"cliente\_id"}\NormalTok{: [}\DecValTok{1}\NormalTok{, }\DecValTok{2}\NormalTok{, }\DecValTok{3}\NormalTok{, }\DecValTok{4}\NormalTok{],}
    \StringTok{"nome"}\NormalTok{: [}\StringTok{"Ana"}\NormalTok{, }\StringTok{"Bruno"}\NormalTok{, }\StringTok{"Clara"}\NormalTok{, }\StringTok{"Daniel"}\NormalTok{]}
\NormalTok{\})}

\BuiltInTok{print}\NormalTok{(clientes)}
\end{Highlighting}
\end{Shaded}

\begin{verbatim}
## shape: (4, 2)
## ┌────────────┬────────┐
## │ cliente_id ┆ nome   │
## │ ---        ┆ ---    │
## │ i64        ┆ str    │
## ╞════════════╪════════╡
## │ 1          ┆ Ana    │
## │ 2          ┆ Bruno  │
## │ 3          ┆ Clara  │
## │ 4          ┆ Daniel │
## └────────────┴────────┘
\end{verbatim}

Dados Compras

\begin{Shaded}
\begin{Highlighting}[]
\NormalTok{pedidos }\OperatorTok{=}\NormalTok{ pl.DataFrame(\{}
    \StringTok{"pedido\_id"}\NormalTok{: [}\DecValTok{101}\NormalTok{, }\DecValTok{102}\NormalTok{, }\DecValTok{103}\NormalTok{, }\DecValTok{104}\NormalTok{, }\DecValTok{105}\NormalTok{],}
    \StringTok{"cliente\_id"}\NormalTok{: [}\DecValTok{1}\NormalTok{, }\DecValTok{2}\NormalTok{, }\DecValTok{3}\NormalTok{, }\DecValTok{1}\NormalTok{, }\DecValTok{5}\NormalTok{],}
    \StringTok{"valor"}\NormalTok{: [}\FloatTok{100.50}\NormalTok{, }\FloatTok{250.75}\NormalTok{, }\FloatTok{75.00}\NormalTok{, }\FloatTok{130.00}\NormalTok{, }\FloatTok{79.00}\NormalTok{]}
\NormalTok{\})}

\BuiltInTok{print}\NormalTok{(pedidos)}
\end{Highlighting}
\end{Shaded}

\begin{verbatim}
## shape: (5, 3)
## ┌───────────┬────────────┬────────┐
## │ pedido_id ┆ cliente_id ┆ valor  │
## │ ---       ┆ ---        ┆ ---    │
## │ i64       ┆ i64        ┆ f64    │
## ╞═══════════╪════════════╪════════╡
## │ 101       ┆ 1          ┆ 100.5  │
## │ 102       ┆ 2          ┆ 250.75 │
## │ 103       ┆ 3          ┆ 75.0   │
## │ 104       ┆ 1          ┆ 130.0  │
## │ 105       ┆ 5          ┆ 79.0   │
## └───────────┴────────────┴────────┘
\end{verbatim}

Exemplo INNER JOIN

\begin{Shaded}
\begin{Highlighting}[]
\NormalTok{res\_ij }\OperatorTok{=}\NormalTok{ clientes.join(pedidos, on}\OperatorTok{=}\StringTok{"cliente\_id"}\NormalTok{, how}\OperatorTok{=}\StringTok{"inner"}\NormalTok{) }\CommentTok{\#Média dos valores comprados por cada cliente.}
\BuiltInTok{print}\NormalTok{(res\_ij)}
\end{Highlighting}
\end{Shaded}

\begin{verbatim}
## shape: (4, 4)
## ┌────────────┬───────┬───────────┬────────┐
## │ cliente_id ┆ nome  ┆ pedido_id ┆ valor  │
## │ ---        ┆ ---   ┆ ---       ┆ ---    │
## │ i64        ┆ str   ┆ i64       ┆ f64    │
## ╞════════════╪═══════╪═══════════╪════════╡
## │ 1          ┆ Ana   ┆ 101       ┆ 100.5  │
## │ 2          ┆ Bruno ┆ 102       ┆ 250.75 │
## │ 3          ┆ Clara ┆ 103       ┆ 75.0   │
## │ 1          ┆ Ana   ┆ 104       ┆ 130.0  │
## └────────────┴───────┴───────────┴────────┘
\end{verbatim}

Exemplo LEFT JOIN

\begin{Shaded}
\begin{Highlighting}[]
\NormalTok{res\_lj }\OperatorTok{=}\NormalTok{ clientes.join(pedidos, on}\OperatorTok{=}\StringTok{"cliente\_id"}\NormalTok{, how}\OperatorTok{=}\StringTok{"left"}\NormalTok{) }
\BuiltInTok{print}\NormalTok{(res\_lj)}
\end{Highlighting}
\end{Shaded}

\begin{verbatim}
## shape: (5, 4)
## ┌────────────┬────────┬───────────┬────────┐
## │ cliente_id ┆ nome   ┆ pedido_id ┆ valor  │
## │ ---        ┆ ---    ┆ ---       ┆ ---    │
## │ i64        ┆ str    ┆ i64       ┆ f64    │
## ╞════════════╪════════╪═══════════╪════════╡
## │ 1          ┆ Ana    ┆ 101       ┆ 100.5  │
## │ 1          ┆ Ana    ┆ 104       ┆ 130.0  │
## │ 2          ┆ Bruno  ┆ 102       ┆ 250.75 │
## │ 3          ┆ Clara  ┆ 103       ┆ 75.0   │
## │ 4          ┆ Daniel ┆ null      ┆ null   │
## └────────────┴────────┴───────────┴────────┘
\end{verbatim}

Exemplo RIGHT JOIN

\begin{Shaded}
\begin{Highlighting}[]
\NormalTok{res\_rj }\OperatorTok{=}\NormalTok{ clientes.join(pedidos, on}\OperatorTok{=}\StringTok{"cliente\_id"}\NormalTok{, how}\OperatorTok{=}\StringTok{"right"}\NormalTok{) }
\BuiltInTok{print}\NormalTok{(res\_rj)}
\end{Highlighting}
\end{Shaded}

\begin{verbatim}
## shape: (5, 4)
## ┌───────┬───────────┬────────────┬────────┐
## │ nome  ┆ pedido_id ┆ cliente_id ┆ valor  │
## │ ---   ┆ ---       ┆ ---        ┆ ---    │
## │ str   ┆ i64       ┆ i64        ┆ f64    │
## ╞═══════╪═══════════╪════════════╪════════╡
## │ Ana   ┆ 101       ┆ 1          ┆ 100.5  │
## │ Bruno ┆ 102       ┆ 2          ┆ 250.75 │
## │ Clara ┆ 103       ┆ 3          ┆ 75.0   │
## │ Ana   ┆ 104       ┆ 1          ┆ 130.0  │
## │ null  ┆ 105       ┆ 5          ┆ 79.0   │
## └───────┴───────────┴────────────┴────────┘
\end{verbatim}

Exemplo OUTER JOIN

\begin{Shaded}
\begin{Highlighting}[]
\NormalTok{res\_oj }\OperatorTok{=}\NormalTok{ clientes.join(pedidos, on}\OperatorTok{=}\StringTok{"cliente\_id"}\NormalTok{, how}\OperatorTok{=}\StringTok{"full"}\NormalTok{)}
\BuiltInTok{print}\NormalTok{(res\_oj)}
\end{Highlighting}
\end{Shaded}

\begin{verbatim}
## shape: (6, 5)
## ┌────────────┬────────┬───────────┬──────────────────┬────────┐
## │ cliente_id ┆ nome   ┆ pedido_id ┆ cliente_id_right ┆ valor  │
## │ ---        ┆ ---    ┆ ---       ┆ ---              ┆ ---    │
## │ i64        ┆ str    ┆ i64       ┆ i64              ┆ f64    │
## ╞════════════╪════════╪═══════════╪══════════════════╪════════╡
## │ 1          ┆ Ana    ┆ 101       ┆ 1                ┆ 100.5  │
## │ 2          ┆ Bruno  ┆ 102       ┆ 2                ┆ 250.75 │
## │ 3          ┆ Clara  ┆ 103       ┆ 3                ┆ 75.0   │
## │ 1          ┆ Ana    ┆ 104       ┆ 1                ┆ 130.0  │
## │ null       ┆ null   ┆ 105       ┆ 5                ┆ 79.0   │
## │ 4          ┆ Daniel ┆ null      ┆ null             ┆ null   │
## └────────────┴────────┴───────────┴──────────────────┴────────┘
\end{verbatim}

Exemplo CROSS JOIN

\begin{Shaded}
\begin{Highlighting}[]
\NormalTok{res\_cj }\OperatorTok{=}\NormalTok{ clientes.join(pedidos, how}\OperatorTok{=}\StringTok{"cross"}\NormalTok{)}
\BuiltInTok{print}\NormalTok{(res\_cj)}
\end{Highlighting}
\end{Shaded}

\begin{verbatim}
## shape: (20, 5)
## ┌────────────┬────────┬───────────┬──────────────────┬────────┐
## │ cliente_id ┆ nome   ┆ pedido_id ┆ cliente_id_right ┆ valor  │
## │ ---        ┆ ---    ┆ ---       ┆ ---              ┆ ---    │
## │ i64        ┆ str    ┆ i64       ┆ i64              ┆ f64    │
## ╞════════════╪════════╪═══════════╪══════════════════╪════════╡
## │ 1          ┆ Ana    ┆ 101       ┆ 1                ┆ 100.5  │
## │ 1          ┆ Ana    ┆ 102       ┆ 2                ┆ 250.75 │
## │ 1          ┆ Ana    ┆ 103       ┆ 3                ┆ 75.0   │
## │ 1          ┆ Ana    ┆ 104       ┆ 1                ┆ 130.0  │
## │ 1          ┆ Ana    ┆ 105       ┆ 5                ┆ 79.0   │
## │ …          ┆ …      ┆ …         ┆ …                ┆ …      │
## │ 4          ┆ Daniel ┆ 101       ┆ 1                ┆ 100.5  │
## │ 4          ┆ Daniel ┆ 102       ┆ 2                ┆ 250.75 │
## │ 4          ┆ Daniel ┆ 103       ┆ 3                ┆ 75.0   │
## │ 4          ┆ Daniel ┆ 104       ┆ 1                ┆ 130.0  │
## │ 4          ┆ Daniel ┆ 105       ┆ 5                ┆ 79.0   │
## └────────────┴────────┴───────────┴──────────────────┴────────┘
\end{verbatim}

P1: Qual é o valor médio das compras realizadas para cada cliente
identificado? Como responder P1?

\begin{Shaded}
\begin{Highlighting}[]
\BuiltInTok{print}\NormalTok{(clientes)}
\end{Highlighting}
\end{Shaded}

\begin{verbatim}
## shape: (4, 2)
## ┌────────────┬────────┐
## │ cliente_id ┆ nome   │
## │ ---        ┆ ---    │
## │ i64        ┆ str    │
## ╞════════════╪════════╡
## │ 1          ┆ Ana    │
## │ 2          ┆ Bruno  │
## │ 3          ┆ Clara  │
## │ 4          ┆ Daniel │
## └────────────┴────────┘
\end{verbatim}

\begin{Shaded}
\begin{Highlighting}[]
\BuiltInTok{print}\NormalTok{(pedidos)}
\end{Highlighting}
\end{Shaded}

\begin{verbatim}
## shape: (5, 3)
## ┌───────────┬────────────┬────────┐
## │ pedido_id ┆ cliente_id ┆ valor  │
## │ ---       ┆ ---        ┆ ---    │
## │ i64       ┆ i64        ┆ f64    │
## ╞═══════════╪════════════╪════════╡
## │ 101       ┆ 1          ┆ 100.5  │
## │ 102       ┆ 2          ┆ 250.75 │
## │ 103       ┆ 3          ┆ 75.0   │
## │ 104       ┆ 1          ┆ 130.0  │
## │ 105       ┆ 5          ┆ 79.0   │
## └───────────┴────────────┴────────┘
\end{verbatim}

Resposta:

\begin{Shaded}
\begin{Highlighting}[]
\NormalTok{res }\OperatorTok{=}\NormalTok{ res\_ij.group\_by([}\StringTok{"nome"}\NormalTok{, }\StringTok{"cliente\_id"}\NormalTok{]).agg(pl.col(}\StringTok{"valor"}\NormalTok{).mean())}
\BuiltInTok{print}\NormalTok{(res)}
\end{Highlighting}
\end{Shaded}

\begin{verbatim}
## shape: (3, 3)
## ┌───────┬────────────┬────────┐
## │ nome  ┆ cliente_id ┆ valor  │
## │ ---   ┆ ---        ┆ ---    │
## │ str   ┆ i64        ┆ f64    │
## ╞═══════╪════════════╪════════╡
## │ Ana   ┆ 1          ┆ 115.25 │
## │ Bruno ┆ 2          ┆ 250.75 │
## │ Clara ┆ 3          ┆ 75.0   │
## └───────┴────────────┴────────┘
\end{verbatim}

P2: Informe os nomes e a quantidade de compras com valor mínimo de
\$100.00 realizadas por cada cliente. Como responder P2?

\begin{Shaded}
\begin{Highlighting}[]
\BuiltInTok{print}\NormalTok{(clientes)}
\end{Highlighting}
\end{Shaded}

\begin{verbatim}
## shape: (4, 2)
## ┌────────────┬────────┐
## │ cliente_id ┆ nome   │
## │ ---        ┆ ---    │
## │ i64        ┆ str    │
## ╞════════════╪════════╡
## │ 1          ┆ Ana    │
## │ 2          ┆ Bruno  │
## │ 3          ┆ Clara  │
## │ 4          ┆ Daniel │
## └────────────┴────────┘
\end{verbatim}

\begin{Shaded}
\begin{Highlighting}[]
\BuiltInTok{print}\NormalTok{(pedidos)}
\end{Highlighting}
\end{Shaded}

\begin{verbatim}
## shape: (5, 3)
## ┌───────────┬────────────┬────────┐
## │ pedido_id ┆ cliente_id ┆ valor  │
## │ ---       ┆ ---        ┆ ---    │
## │ i64       ┆ i64        ┆ f64    │
## ╞═══════════╪════════════╪════════╡
## │ 101       ┆ 1          ┆ 100.5  │
## │ 102       ┆ 2          ┆ 250.75 │
## │ 103       ┆ 3          ┆ 75.0   │
## │ 104       ┆ 1          ┆ 130.0  │
## │ 105       ┆ 5          ┆ 79.0   │
## └───────────┴────────────┴────────┘
\end{verbatim}

Reposta:

\begin{Shaded}
\begin{Highlighting}[]
\NormalTok{res }\OperatorTok{=}\NormalTok{ (res\_oj.with\_columns(pl.col(}\StringTok{"valor"}\NormalTok{) }\OperatorTok{\textgreater{}} \DecValTok{100}\NormalTok{)}
\NormalTok{       .group\_by(}\StringTok{"nome"}\NormalTok{)}
\NormalTok{       .agg(pl.col(}\StringTok{"valor"}\NormalTok{).}\BuiltInTok{sum}\NormalTok{())) }\CommentTok{\#\#transforma os valores maiores que 100 e depois conta por cliente}
\BuiltInTok{print}\NormalTok{(res)}
\end{Highlighting}
\end{Shaded}

\begin{verbatim}
## shape: (5, 2)
## ┌────────┬───────┐
## │ nome   ┆ valor │
## │ ---    ┆ ---   │
## │ str    ┆ u32   │
## ╞════════╪═══════╡
## │ null   ┆ 0     │
## │ Bruno  ┆ 1     │
## │ Ana    ┆ 2     │
## │ Daniel ┆ 0     │
## │ Clara  ┆ 0     │
## └────────┴───────┘
\end{verbatim}

JOIN com Múltiplas Colunas como Chave

\begin{Shaded}
\begin{Highlighting}[]
\NormalTok{vendas }\OperatorTok{=}\NormalTok{ pl.DataFrame(\{}
    \StringTok{"id\_venda"}\NormalTok{: [}\DecValTok{1}\NormalTok{, }\DecValTok{2}\NormalTok{, }\DecValTok{3}\NormalTok{],}
    \StringTok{"id\_cl"}\NormalTok{: [}\DecValTok{1}\NormalTok{, }\DecValTok{2}\NormalTok{, }\DecValTok{1}\NormalTok{],}
    \StringTok{"id\_prod"}\NormalTok{: [}\DecValTok{101}\NormalTok{, }\DecValTok{102}\NormalTok{, }\DecValTok{103}\NormalTok{],}
    \StringTok{"qtde"}\NormalTok{: [}\DecValTok{2}\NormalTok{, }\DecValTok{1}\NormalTok{, }\DecValTok{1}\NormalTok{]}
\NormalTok{\})}

\NormalTok{detalhes\_pedidos }\OperatorTok{=}\NormalTok{ pl.DataFrame(\{}
    \StringTok{"id\_ped"}\NormalTok{: [}\DecValTok{201}\NormalTok{, }\DecValTok{202}\NormalTok{, }\DecValTok{203}\NormalTok{],}
    \StringTok{"cl\_id"}\NormalTok{: [}\DecValTok{1}\NormalTok{, }\DecValTok{2}\NormalTok{, }\DecValTok{1}\NormalTok{],}
    \StringTok{"id\_prod"}\NormalTok{: [}\DecValTok{101}\NormalTok{, }\DecValTok{102}\NormalTok{, }\DecValTok{104}\NormalTok{],}
    \StringTok{"valor"}\NormalTok{: [}\FloatTok{50.00}\NormalTok{, }\FloatTok{75.00}\NormalTok{, }\FloatTok{100.00}\NormalTok{]}
\NormalTok{\})}

\BuiltInTok{print}\NormalTok{(vendas)}
\end{Highlighting}
\end{Shaded}

\begin{verbatim}
## shape: (3, 4)
## ┌──────────┬───────┬─────────┬──────┐
## │ id_venda ┆ id_cl ┆ id_prod ┆ qtde │
## │ ---      ┆ ---   ┆ ---     ┆ ---  │
## │ i64      ┆ i64   ┆ i64     ┆ i64  │
## ╞══════════╪═══════╪═════════╪══════╡
## │ 1        ┆ 1     ┆ 101     ┆ 2    │
## │ 2        ┆ 2     ┆ 102     ┆ 1    │
## │ 3        ┆ 1     ┆ 103     ┆ 1    │
## └──────────┴───────┴─────────┴──────┘
\end{verbatim}

\begin{Shaded}
\begin{Highlighting}[]
\BuiltInTok{print}\NormalTok{(detalhes\_pedidos)}
\end{Highlighting}
\end{Shaded}

\begin{verbatim}
## shape: (3, 4)
## ┌────────┬───────┬─────────┬───────┐
## │ id_ped ┆ cl_id ┆ id_prod ┆ valor │
## │ ---    ┆ ---   ┆ ---     ┆ ---   │
## │ i64    ┆ i64   ┆ i64     ┆ f64   │
## ╞════════╪═══════╪═════════╪═══════╡
## │ 201    ┆ 1     ┆ 101     ┆ 50.0  │
## │ 202    ┆ 2     ┆ 102     ┆ 75.0  │
## │ 203    ┆ 1     ┆ 104     ┆ 100.0 │
## └────────┴───────┴─────────┴───────┘
\end{verbatim}

Realizando um JOIN com Múltiplas Colunas

\begin{Shaded}
\begin{Highlighting}[]
\NormalTok{final }\OperatorTok{=}\NormalTok{ vendas.join(detalhes\_pedidos,}
\NormalTok{                    left\_on }\OperatorTok{=}\NormalTok{ [}\StringTok{"id\_cl"}\NormalTok{, }\StringTok{"id\_prod"}\NormalTok{],}
\NormalTok{                    right\_on }\OperatorTok{=}\NormalTok{ [}\StringTok{"cl\_id"}\NormalTok{, }\StringTok{"id\_prod"}\NormalTok{],}
\NormalTok{                    how }\OperatorTok{=} \StringTok{"inner"}\NormalTok{) }\CommentTok{\#Faz o join usando duas chaves ao mesmo tempo (id\_cl + id\_prod). Assim só bate quem tiver cliente e produto iguais nos dois DataFrames.}
\BuiltInTok{print}\NormalTok{(final)}
\end{Highlighting}
\end{Shaded}

\begin{verbatim}
## shape: (2, 6)
## ┌──────────┬───────┬─────────┬──────┬────────┬───────┐
## │ id_venda ┆ id_cl ┆ id_prod ┆ qtde ┆ id_ped ┆ valor │
## │ ---      ┆ ---   ┆ ---     ┆ ---  ┆ ---    ┆ ---   │
## │ i64      ┆ i64   ┆ i64     ┆ i64  ┆ i64    ┆ f64   │
## ╞══════════╪═══════╪═════════╪══════╪════════╪═══════╡
## │ 1        ┆ 1     ┆ 101     ┆ 2    ┆ 201    ┆ 50.0  │
## │ 2        ┆ 2     ┆ 102     ┆ 1    ┆ 202    ┆ 75.0  │
## └──────────┴───────┴─────────┴──────┴────────┴───────┘
\end{verbatim}

\end{document}
